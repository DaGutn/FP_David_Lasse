\newpage
\section{Auswertung}
    \subsection{Vorbereitungsmessungen}
        Für die erste Vorbereitungsmessung werden beispielhaft die Messungen für einen Zylinder, einer Reihe aus vier Zylindern und einer Reihe aus acht Zylindern der Länge \SI{50}{\milli\metre} mit den 
        entsprechenden Oszilloskopmessungen verglichen. Die Abbildungen~\ref{fig:pre_vgl} zeigen, dass die Messungen per Computersoftware und Oszilloskop weitgehend übereinstimmen. Die Peaks und deren 
        relativen Höhen auf dem Oszilloskop spiegeln die Darstellung der Computersoftware wieder, werden jedoch weniger scharf dargestellt. Deswegen ist es sinvoll die Computermessungen zur weiteren Analyse
        zu nutzen. 
        \begin{figure}
            \centering
            \begin{subfigure}[b]{0.45\textwidth}
                \centering
                \includegraphics[scale=0.4]{./pictures/1_Zylinder_50mm.pdf}
                \caption{In der Abbildung ist das Frequenzspektrum eines einzelnen Hohlraumresonators der Länge \SI{50}{\milli\metre} per Computerprogramm dargestellt.}
                %\label{fig:e_feld_chirp}
            \end{subfigure}
            \hfill
            \centering
            \begin{subfigure}[b]{0.45\textwidth}
                \centering
                \includegraphics[scale=0.13]{./pictures/1_Zylinder.jpg}
                \caption{In der Abbildung ist das Frequenzspektrum eines einzelnen Hohlraumresonators der Länge \SI{50}{\milli\metre} per Oszilloskop dargestellt.}
                %\label{fig:chirp}
            \end{subfigure}

            \centering
            \begin{subfigure}[b]{0.45\textwidth}
                \centering
                \includegraphics[scale=0.4]{./pictures/4_Zylinder_50mm.pdf}
                \caption{In der Abbildung ist das Frequenzspektrum einer Kette aus vier Hohlraumresonators der Länge \SI{50}{\milli\metre} per Computerprogramm dargestellt.}
                %\label{fig:e_feld_chirp}
            \end{subfigure}
            \hfill
            \centering
            \begin{subfigure}[b]{0.45\textwidth}
                \centering
                \includegraphics[scale=0.13]{./pictures/4_Zylinder.jpg}
                \caption{In der Abbildung ist das Frequenzspektrum einer Kette aus vier Hohlraumresonators der Länge \SI{50}{\milli\metre} per Oszilloskop dargestellt.}
                %\label{fig:chirp}
            \end{subfigure}

            \centering
            \begin{subfigure}[b]{0.45\textwidth}
                \centering
                \includegraphics[scale=0.4]{./pictures/8_Zylinder_50mm.pdf}
                \caption{In der Abbildung ist das Frequenzspektrum einer Kette aus acht Hohlraumresonators der Länge \SI{50}{\milli\metre} per Computerprogramm dargestellt.}
                %\label{fig:e_feld_chirp}
            \end{subfigure}
            \hfill
            \centering
            \begin{subfigure}[b]{0.45\textwidth}
                \centering
                \includegraphics[scale=0.13]{./pictures/8_Zylinder.jpg}
                \caption{In der Abbildung ist das Frequenzspektrum einer Kette aus acht Hohlraumresonators der Länge \SI{50}{\milli\metre} per Oszilloskop dargestellt.}
                %\label{fig:chirp}
            \end{subfigure}
            \caption{}
            \label{fig:pre_vgl}
        \end{figure}
    \FloatBarrier

    \subsection{Wasserstoffatom}
        \subsubsection*{Resonanzfrequenzen bei $\theta$=180°}
            Zunächst wird das gemessene hochaufgelöste Frequenzspektrum des Hohlraumresonators bei einem Winkel von $\theta=\SI{180}{\degree}$ und einem Frequenzbereich zwischen 0,1 und \SI{12}{\kilo\hertz}
            in Abbildung \ref{fig:hatom_180} dargestellt.
            \begin{figure}[ht!]
                \centering
                \includegraphics[scale=0.6]{./pictures/hatom_180.pdf}
                \caption{In der Abbildung ist das hochaufgelöste Spektrum des Wasserstoffatommodells bei einem Polarwinkkel $\theta$ von 180° im Bereich von 0,1 bis \SI{12}{\kilo\hertz} dargestellt.}
                \label{fig:hatom_180}
            \end{figure}

            \FloatBarrier
            Mit Hilfe des Oszilloskops werden die in Abbildung \ref{fig:hatom_180} zu sehenden Resonanzfrequenzen analysiert und ihre Ordnung, Amplitude und Phasenverschiebung in Tabelle~\ref{tab:phasenverschiebung} 
            aufgelistet.
            \FloatBarrier
        \newpage
        \subsubsection*{Winkelabhängigkeit gewählter Resonanzfrequenzen}
            Die Druckamplitude der Resonanzfrequenzen wird für die Resonanzfrequenzen \SI{2.310}{\kilo\hertz}, \SI{3.711}{\kilo\hertz}, \SI{4.999}{\kilo\hertz} und \SI{7.470}{\kilo\hertz} in Abhängigkeit des 
            Winkels $\theta$ aufgetragen. Die in den Abbildungen \ref{fig:H_atom_resonanz_1_2310Hz} zu sehende Druckamplitudenverteilung kann unterschiedlich interpretiert werden.Deutlich ist eine Keule in 
            180°-Richtung zu erkennen. ZUsätzlich ist auch ein sehr schwach ausgeprägter Ausschlag in 90°-Richtung zu erkennen. Je nach Bewertung des Ausschlags in 90°-Richtung kann von einem 3d$_0$- oder 
            2p$_0$-Orbital ausgegangen werden. Die in Abbildung \ref{fig:H_atom_resonanz_2_3711Hz} zu sehende Druckamplitudenverteilung weißt ebenfalls eine große Keule in 180°-Richtung und eine Keule in 
            90°-Richtung, die sich womöglich aus zwei kleinen Keulen zusammensetzt auf. Wird davon ausgegangen, dass nur eine eine gesamte Keule in 90°-Richtung existiert, kann davon ausgegangen werden, dass es 
            sich um das Äquivalent eines 3d$_0$-Orbitals handelt. Die in Abbildung \ref{fig:H_atom_resonanz_2_3711Hz} zu sehende Druckamplitudenverteilungen könnte auf ein 4f$_0$-Orbital hinweisen, sofern das 
            eine Maximum nahe 90° vernachlässigt und so eine Keule erkennbar ist. In Abbildung \ref{fig:H_atom_resonanz_4_7470Hz} scheint das Äquivalent eines 6h$_0$-Orbitals aufgenommen worden zu sein. Zur 
            Bestimmung der Orbitale wurden die gemessenen Verteilungen mit theoretischen Atomorbitalen verglichen~\cite{geek3_atomic_2018}.
            \FloatBarrier
            \begin{figure}[ht]
                \centering
                \begin{subfigure}[b]{0.45\textwidth}
                    \centering
                    \includegraphics[scale=0.3]{./pictures/H_atom_resonanz_1_2310Hz.pdf}
                    \caption{In der Abbildung Druckamplitude an der Resonanzstelle bei \SI{2.310}{\kilo\hertz} gegen den Polarwinkel $\theta$ aufgetragen. Die Druckamplitudenverteilung lässt ein 3d$_0$- oder 2p$_0$-Orbital vermuten.}
                    \label{fig:H_atom_resonanz_1_2310Hz}
                \end{subfigure}
                \hfill
                \centering
                \begin{subfigure}[b]{0.45\textwidth}
                    \centering
                    \includegraphics[scale=0.3]{./pictures/H_atom_resonanz_2_3711Hz.pdf}
                    \caption{In der Abbildung Druckamplitude an der Resonanzstelle bei \SI{3.711}{\kilo\hertz} gegen den Polarwinkel $\theta$ aufgetragen. Die Druckamplitudenverteilung lässt ein 3d$_0$-Orbital vermuten.}
                    \label{fig:H_atom_resonanz_2_3711Hz}
                \end{subfigure}
            \end{figure}
            \begin{figure}
                \centering
                \begin{subfigure}[b]{0.45\textwidth}
                    \centering
                    \includegraphics[scale=0.3]{./pictures/H_atom_resonanz_3_4999Hz.pdf}
                    \caption{In der Abbildung Druckamplitude an der Resonanzstelle bei \SI{4.999}{\kilo\hertz} gegen den Polarwinkel $\theta$ aufgetragen. Die Druckamplitudenverteilung lässt ein 4f$_0$-Orbital vermuten.}
                    \label{fig:H_atom_resonanz_3_4999Hz}
                \end{subfigure}
                \hfill
                \centering
                \begin{subfigure}[b]{0.45\textwidth}
                    \centering
                    \includegraphics[scale=0.3]{./pictures/H_atom_resonanz_4_7470Hz.pdf}
                    \caption{In der Abbildung Druckamplitude an der Resonanzstelle bei \SI{7.470}{\kilo\hertz} gegen den Polarwinkel $\theta$ aufgetragen. Die Druckamplitudenverteilung lässt ein 6h$_0$-Orbital vermuten.}
                    \label{fig:H_atom_resonanz_4_7470Hz}
                \end{subfigure}
            \end{figure}
              
        \newpage
        \FloatBarrier
        \subsubsection*{Zustandsaufspaltung}
            Nach dem Einsetzten von Zwischenringen ergeben sich Aufspaltungen der Resonanz um \SI{2.3}{\kilo\hertz}, die in Abbildung \ref{fig:hatom_180_3mm} exemplarisch für eine Ringlänge von 
            \SI{3}{\milli\metre} eingezeichnet ist. Da die Resonanz in zwei Resonanzen aufspaltet, muss die Drehimpulsquantenzahl gleich eins sein. So kann nun defintiv von einem 2p$_0$-Orbital
            ausgegangen werden.
            \FloatBarrier 
            \begin{figure}[ht]
                \centering
                \includegraphics[scale=0.5]{./pictures/hatom_180_3mm.pdf}
                \caption{In der Abbildung ist die Aufspaltung der Resonanz bei \SI{2.3}{\kilo\hertz} in zwei circa \SI{60}{\hertz} entfernte Resonanzen deutlich zu erkennen.}
                \label{fig:hatom_180_3mm}
            \end{figure}
            \FloatBarrier
            Die Aufspaltung der Resonanzfrequenzen steigt, wie in Abbildung~\ref{fig:f_Aufspaltung} zu sehen, linear mit der Länge des Zwischenrings auf einen maximalen Wert von \SI{171}{\hertz} an.
            Dies entspicht der zum Magnetfeld proportionalen Zeeman-Aufspaltung eines Wasserstoffatoms in einem externen Magnetfeld.
            \FloatBarrier
            \begin{figure}[ht]
                \centering
                \includegraphics[scale=0.6]{./pictures/f_Aufspaltung.pdf}
                \caption{Aus der Abbildung ist ein linearer Zusammenhang zwischen der Aufspaltung der Resonanzzustände und der Länge des eingesetzten Zwischenrings zu vermuten.}
                \label{fig:f_Aufspaltung}
            \end{figure}
        \newpage
        \subsubsection*{Winkelabhängigkeit der Zustandsaufspaltung}
            Die Winkelabhängigkeit der Resonanzfrequenzaufspaltung ist für einen Zwischenring der Länge \SI{9}{\milli\metre} in einem Polarplot \ref{fig:H_atom_resonanz_4} aufgetragen.  
            \FloatBarrier 
            \begin{figure}[ht!]
                \centering
                \includegraphics[scale=0.5]{./pictures/H_atom_resonanz_4.pdf}
                \caption{In der Abbildung ist die Druckamplitude gegen den Winkel $\theta$ aufgetragen.}
                \label{fig:H_atom_resonanz_4}
            \end{figure}
            \FloatBarrier
            Die Aufspaltung ist in Abbildung~\ref{fig:hatom_180_9mm} exemplarisch für einen Winkel von 180° dargestellt. Die Resonanzfrequenz bei \SI{2.170}{\kilo\hertz} entspricht einem Zustand mit den 
            Quantenzahlen m=0 und l=1 . Die Resonanzfrequenz bei \SI{2.284}{\kilo\hertz} entspricht dem Zustand mit den Quantenzahlen m=$\pm 1$ und l=1. Diese Aufspaltung ist wieder das Äquivalent des
            Zeeman-Effekts, der in diesem Fall der Änderung einer Symmetrieachse, durch das Verlängern des Resonators in eine Richtung entspricht.
            \FloatBarrier
            \begin{figure}[ht]
                \centering
                \includegraphics[scale=0.5]{./pictures/hatom_180_9mm.pdf}
                \caption{In der Abbildung ist die Aufspaltung der Resonanz exemplarisch für einen Winkel von 180° aufgetragen.}
                \label{fig:hatom_180_9mm}
            \end{figure}
    \newpage
    \subsection{Wasserstoffmolekül}
        \subsubsection*{Resonanzfrequenzen für verschiedene Blendendurchmesser}
            Für die verschiedenen Blendendurchmesser \SI{10}{\milli\metre}, \SI{15}{\milli\metre} und \SI{20}{\milli\metre} ergeben sich im Frequenzbereich von 2,2 bis \SI{2.5}{\kilo\hertz} drei 
            Resonanzfrequenzen, deren exakte Werte in Abhängigkeit vom Blendendurchmesser in Abbildung \ref{fig:res_freq_gegen_d_blende} aufgetragen sind. Es ist deutlich zu erkennen, dass nur die dritte
            Resonanzfrequenz vom Blendendurchmesser abhängt. 
            \begin{figure}[ht]
                \centering
                \includegraphics[scale=0.6]{./pictures/res_freq_gegen_d_blende.pdf}
                \caption{In der Abbildung sind die drei Resonanzfrequenzen im Bereich von 2,2 bis \SI{2.5}{\kilo\hertz} gegen den Durchmesser der eingesetzten Blende dargestellt.}
                \label{fig:res_freq_gegen_d_blende}
            \end{figure}
        \FloatBarrier
        \newpage
        \subsubsection*{Winkelabhängigkeit gewählter Resonanzfrequenzen}
            Während für einen Blendendurchmesser von \SI{16}{\milli\metre} maximal vier Resonanzen zu erwarten sind, werden wie in Abbildung \ref{fig:hmol_res} zu sehen nur die Resonanzfrequenzen 
            \SI{2307}{\hertz}, \SI{2313}{\hertz} und \SI{2426}{\hertz} gemessen.  
            \begin{figure}[ht]
                \centering
                \includegraphics[scale=0.5]{./pictures/hmol_res.pdf}
                \caption{In der Abbildung ist das Frequenzspektrum für eine Winkel von 180° aufgetragen. Es sind Resonanzen bei \SI{2307}{\hertz}, \SI{2313}{\hertz} und \SI{2426}{\hertz} zu erkennen.}
                \label{fig:hmol_res}
            \end{figure}
            \FloatBarrier
            Für diese drei Resonanzfrequenzen werden die Druckamplituden in Abhängigkeit vom Winkel $\varphi$ in Polarplots aufgetragen 

            \FloatBarrier
            \begin{figure}[ht]
                \centering
                \begin{subfigure}[b]{0.45\textwidth}
                    \centering
                    \includegraphics[scale=0.3]{./pictures/H_mol_resonanz_1_2307Hz.pdf}
                    \caption{In der Abbildung ist die Druckamplitude der Resonanzstelle bei \SI{2307}{\hertz} gegen den Winkel $\varphi$ aufgetragen.}
                    \label{fig:H_mol_resonanz_1_2307Hz}
                \end{subfigure}
                \hfill
                \centering
                \begin{subfigure}[b]{0.45\textwidth}
                    \centering
                    \includegraphics[scale=0.3]{./pictures/H_mol_resonanz_1_2313Hz.pdf}
                    \caption{In der Abbildung ist die Druckamplitude der Resonanzstelle bei \SI{2313}{\hertz} gegen den Winkel $\varphi$ aufgetragen.}
                    \label{fig:H_mol_resonanz_1_2313Hz}
                \end{subfigure}

                \centering
                \begin{subfigure}[b]{0.45\textwidth}
                    \centering
                    \includegraphics[scale=0.3]{./pictures/H_mol_resonanz_1_2426Hz.pdf}
                    \caption{In der Abbildung ist die Druckamplitude der Resonanzstelle bei \SI{2426}{\hertz} gegen den Winkel $\varphi$ aufgetragen.}
                    \label{fig:H_mol_resonanz_1_2426Hz}
                \end{subfigure}
            \end{figure}
            \FloatBarrier
            und die Phasenverschiebungen bei einem Winkel von 180° zwischen der Druckamplitude im oberen und unteren Teil des Resonators aufgelistet~\ref{tab:rel_phasenverschiebung}.
            \FloatBarrier
            \begin{table}[h]
                \centering
                \caption{In dieser Tabelle sind für die drei gemessenen Resonanzfrequenzen $f_{\text{res}}$, die zugehörigen Phasendifferenzen für die obere und die untere Druckamplitude, sowie deren relative Phasendifferenz aufgelistet.}
                \label{tab:rel_phasenverschiebung}
                \begin{tabular}{c c c c c}
                \toprule
                {Ordnung} & {$f_{\text{res}}$ [$\si{\kilo\hertz}]$} & {Phasendiff. oben [°]} & {Phasendiff. unten [°]} & {relative Phasendiff. [°]} \\
                \midrule
                \num{1}  & \num{2.307}  &  \num{-109}  &  \num{-36}   & \num{73}      \\
                \num{2}  & \num{2.313}  &  \num{-103}  &  \num{-85}   & \num{188}     \\
                \num{3}  & \num{2.426}  &  \num{36}    &  \num{-145}  & \num{181}     \\
                \bottomrule
                \end{tabular}
            \end{table}
            Aus den Phasenverschiebung lassen sich die Symmetrien der Zustände feststellen. Bei einer relativen Phasendifferenz von 0° handelt es sich um einen symmetrischen und bei 180° um einen 
            antisymmetrischen Zustand. Demnach handelt es sich bei den Resonanzen bei 2,313 und \SI{2.426}{\kilo\hertz} eindeutig um antisymmetrische Zustände. Da die Verteilung bei \SI{2.426}{\kilo\hertz} 
            unabhängig vom Polarwinkel $\varphi$ ist und die Resonanz bei \SI{2.3}{\kilo\hertz} einem Zustand mit l=1 entsprach, kann von einem $2\sigma_{\text{u}}$-Zustand ausgegangen werden. Die Verteilung 
            bei \SI{2.313}{\kilo\hertz} ist zwar nicht offensichtlich unabhägngig vom Winkel, aber zumindest kreisförmig. Wenn daher von einer Winkelunabhängigkeit, die nur nicht genau genug gemessen wurde, 
            ausgegangen wird, handelt es sich um den $1\sigma_{\text{u}}$-Zustand. Die relative Phasenverschiebung der Resonanz bei \SI{2.307}{\kilo\hertz} liegt mit 73° weit von 0° und 180° 
            entfernt. Da sie etwas näher an 0° liegt und die Verteilung winkelunabhängig ist, wird von einem $1\sigma_{\text{g}}$-Zustand ausgegangen. 

    \newpage
    \subsection{Ein-Dimensionaler Festkörper}
        \subsubsection*{Resonatorkette mit 16mm-Blenden}
            Die aufgenommenen Frequenzspektren der Resonatorketten zeigen jeweils vier Bereiche, die mit Maxima gefüllt sind. Wie in den 
            ~\autoref{fig:jucktnicht} zu erkennen, entspricht die Anzahl der Maxima der der verwendeten Zylinder. Dies lässt sich so 
            interpretieren, dass durch jeden Zylinder ein Band hinzugefügt wird, auf dem Elektronen in einem Festkörper Zustände einnehmen können. Die Bereiche ohne Maxima entsprechen Bandlücken, also Bereichen,
            in denen keine Elektronenzustände vorliegen.     
            
            \FloatBarrier
            \begin{figure}[ht]
                \centering
                \begin{subfigure}[b]{0.45\textwidth}
                    \centering
                    \includegraphics[scale=0.45]{./pictures/1dim_2_Zylinder_16mm.pdf}
                    \caption{In der Abbildung ist das Frequenzspektrum für eine Resonatorkette aus zwei Zylindern der Länge \SI{50}{\milli\metre}, die durch Blenden mit einem Durchmesser von \SI{16}{\milli\metre} getrennt sind, dargestellt.}
                    \label{fig:1dim_2_Zylinder_16mm}
                \end{subfigure}
                \hfill
                \centering
                \begin{subfigure}[b]{0.45\textwidth}
                    \centering
                    \includegraphics[scale=0.45]{./pictures/1dim_4_Zylinder_16mm.pdf}
                    \caption{In der Abbildung ist das Frequenzspektrum für eine Resonatorkette aus vier Zylindern der Länge \SI{50}{\milli\metre}, die durch Blenden mit einem Durchmesser von \SI{16}{\milli\metre} getrennt sind, dargestellt.}
                    \label{fig:1dim_4_Zylinder_16mm}
                \end{subfigure}

                \centering
                \begin{subfigure}[b]{0.45\textwidth}
                    \centering
                    \includegraphics[scale=0.45]{./pictures/1dim_10_Zylinder_16mm.pdf}
                    \caption{In der Abbildung ist das Frequenzspektrum für eine Resonatorkette aus zehn Zylindern der Länge \SI{50}{\milli\metre}, die durch Blenden mit einem Durchmesser von \SI{16}{\milli\metre} getrennt sind, dargestellt.}
                    \label{fig:1dim_10_Zylinder_16mm}
                \end{subfigure}
                \caption{}
                \label{fig:jucktnicht}
            \end{figure}
            \FloatBarrier


        \subsubsection*{Resonatorkette mit 10mm- und 13mm-Blenden}
            Ein kleinerer Blendendurchmesser entspricht einem größeren Potential innerhalb des Festkörpers. Demnach sind die Elektronen stärker lokalisiert und ihre Bänder flacher. Dies ist in den 
            Abbildungen~\ref{10mm_16mm_blende}\ref{10mm_16mm_blende2} zu erkennen. Besonders für die Frequenzspektren bei einer Blendenwahl von \SI{10}{\milli\metre} sind die kleineren Abstände zwischen den Maxima und die größeren
            Abstände zwischen den Maximabereichen im Vergleich zu den Frequenzspektren bei einer Blendenwahl von \SI{16}{\milli\metre} deutlich zu erkennen.
            \FloatBarrier
            \begin{figure}[ht]
                \centering
                \begin{subfigure}[b]{0.45\textwidth}
                    \centering
                    \includegraphics[scale=0.45]{./pictures/1dim_2_Zylinder_10mm.pdf}
                    \caption{In der Abbildung ist das Frequenzspektrum für eine Resonatorkette aus zwei Zylindern der Länge \SI{50}{\milli\metre}, die durch Blenden mit einem Durchmesser von \SI{10}{\milli\metre} getrennt sind, dargestellt.}
                    \label{fig:1dim_2_Zylinder_10mm}
                \end{subfigure}
                %\hfill
                \centering
                \begin{subfigure}[b]{0.45\textwidth}
                    \centering
                    \includegraphics[scale=0.45]{./pictures/1dim_2_Zylinder_13mm.pdf}
                    \caption{In der Abbildung ist das Frequenzspektrum für eine Resonatorkette aus zwei Zylindern der Länge \SI{50}{\milli\metre}, die durch Blenden mit einem Durchmesser von \SI{13}{\milli\metre} getrennt sind, dargestellt.}
                    \label{fig:1dim_2_Zylinder_13mm}
                \end{subfigure}
                \caption{Auf der linken Seite sind die gemessenen Frequenzspektren für einen Blendendurchmesser von \SI{10}{\milli\metre} und einer Anzahl von zwei Zylindern dargestellt. Auf der rechten Seite sind die entsprechenden Frequenzspektren für einen Blendendurchmesser von \SI{13}{\milli\metre} dargestellt.}
                \label{10mm_16mm_blende}
            \end{figure}
            \FloatBarrier
            \begin{figure}
                %\hfill
                \centering
                \begin{subfigure}[b]{0.45\textwidth}
                    \centering
                    \includegraphics[scale=0.45]{./pictures/1dim_4_Zylinder_10mm.pdf}
                    \caption{In der Abbildung ist das Frequenzspektrum für eine Resonatorkette aus vier Zylindern der Länge \SI{50}{\milli\metre}, die durch Blenden mit einem Durchmesser von \SI{10}{\milli\metre} getrennt sind, dargestellt.}
                    \label{fig:1dim_4_Zylinder_10mm}
                \end{subfigure}
                \centering
                \begin{subfigure}[b]{0.45\textwidth}
                    \centering
                    \includegraphics[scale=0.45]{./pictures/1dim_4_Zylinder_13mm.pdf}
                    \caption{In der Abbildung ist das Frequenzspektrum für eine Resonatorkette aus vier Zylindern der Länge \SI{50}{\milli\metre}, die durch Blenden mit einem Durchmesser von \SI{13}{\milli\metre} getrennt sind, dargestellt.}
                    \label{fig:1dim_4_Zylinder_13mm}
                \end{subfigure}
                %\hfill
                \centering
                \begin{subfigure}[b]{0.45\textwidth}
                    \centering
                    \includegraphics[scale=0.45]{./pictures/1dim_10_Zylinder_10mm.pdf}
                    \caption{In der Abbildung ist das Frequenzspektrum für eine Resonatorkette aus zehn Zylindern der Länge \SI{50}{\milli\metre}, die durch Blenden mit einem Durchmesser von \SI{10}{\milli\metre} getrennt sind, dargestellt.}
                    \label{fig:1dim_10_Zylinder_10mm}
                \end{subfigure}
                %\hfill
                \centering
                \begin{subfigure}[b]{0.45\textwidth}
                    \centering
                    \includegraphics[scale=0.45]{./pictures/1dim_10_Zylinder_13mm.pdf}
                    \caption{In der Abbildung ist das Frequenzspektrum für eine Resonatorkette aus zehn Zylindern der Länge \SI{50}{\milli\metre}, die durch Blenden mit einem Durchmesser von \SI{13}{\milli\metre} getrennt sind, dargestellt.}
                    \label{fig:1dim_10_Zylinder_13mm}
                \end{subfigure}
                \caption{Auf der linken Seite sind die gemessenen Frequenzspektren für einen Blendendurchmesser von \SI{10}{\milli\metre} und die Zylinderanzahlen vier und 10 dargestellt. Auf der rechten Seite sind die entsprechenden Frequenzspektren für einen Blendendurchmesser von \SI{13}{\milli\metre} dargestellt.}
                \label{10mm_16mm_blende2}
            \end{figure}
            \FloatBarrier


        \subsubsection*{Resonatorkette mit Störstellen}
            In Form von Zylindern abweichender Größe werden Störstellen erzeugt, die Gitterdefekten in Festkörpern entspechen simulieren sollen. Diese Störstellen rufen Abweichungen vom ungestörten 
            Frequenzspektrum hervor. So lässt sich für die Störzylinder der Größen \SI{62.5}{\milli\metre} (siehe Abb.~\ref{fig:1dim_4_Zylinder_625_Fehlstelle}) und 
            \SI{75}{\milli\metre} (siehe Abb.~\ref{fig:1dim_10_Zylinder_750_Fehlstelle}) eine neue Resonanz um eine Frequenz von \SI{3}{\kilo\hertz} erkennen. Zudem ist, wie in Abbildung~\ref{fig:Störstelle} zu sehen, für 
            alle Störzylinder eine deutliche Reduktion der Druckamplituden im Vergleich zur ungestörten Resonatorkette aus zehen Zylindern zu erkennen. Auch ein Absacken der Druckamplituden in der Mitte des 
            ersten Maximabereichs zwischen 0 und \SI{3}{\kilo\hertz} ist für alle Störzylinderlängen gut zu erkennen.
            \FloatBarrier
            \begin{figure}[ht]
                \centering
                \begin{subfigure}[b]{0.45\textwidth}
                    \centering
                    \includegraphics[scale=0.45]{./pictures/1dim_10_Zylinder_375_Fehlstelle.pdf}
                    \caption{In der Abbildung ist das Frequenzspektrum für eine Resonatorkette aus neun Zylindern der Länge \SI{50}{\milli\metre} und einem Zylinder der Länge \SI{37.5}{\milli\metre}, die durch Blenden mit einem Durchmesser von \SI{13}{\milli\metre} getrennt sind, dargestellt.}
                    \label{fig:1dim_10_Zylinder_375_Fehlstelle}
                \end{subfigure}
                \hfill
                \centering
                \begin{subfigure}[b]{0.45\textwidth}
                    \centering
                    \includegraphics[scale=0.45]{./pictures/1dim_4_Zylinder_625_Fehlstelle.pdf}
                    \caption{In der Abbildung ist das Frequenzspektrum für eine Resonatorkette aus neun Zylindern der Länge \SI{50}{\milli\metre} und einem Zylinder der Länge \SI{62.5}{\milli\metre}, die durch Blenden mit einem Durchmesser von \SI{13}{\milli\metre} getrennt sind, dargestellt.}
                    \label{fig:1dim_4_Zylinder_625_Fehlstelle}
                \end{subfigure}
                \centering
                \begin{subfigure}[b]{0.45\textwidth}
                    \centering
                    \includegraphics[scale=0.45]{./pictures/1dim_10_Zylinder_750_Fehlstelle.pdf}
                    \caption{In der Abbildung ist das Frequenzspektrum für eine Resonatorkette aus neun Zylindern der Länge \SI{50}{\milli\metre} und einem Zylinder der Länge \SI{75}{\milli\metre}, die durch Blenden mit einem Durchmesser von \SI{13}{\milli\metre} getrennt sind, dargestellt.}
                    \label{fig:1dim_10_Zylinder_750_Fehlstelle}
                \end{subfigure}
                \caption{ }
                \label{fig:Störstelle}
            \end{figure}
            \FloatBarrier

        \subsubsection*{Kette wechselnder Resonatorlängen}
            Die Kette aus abwechselnden 50 und \SI{75}{\milli\metre} langen Resonatoren entspricht einem Gitter mit zwei-atomiger Basis. Dies führt dazu, dass die Resonanzen der einzelnen ein-atomigen Basen
            in die zwei-atomigen Basis  übernommen werden. Dies ist im Vergleich der einzelnen Spektren eines \SI{50}{\milli\metre}-Resonators (siehe Abb.~\ref{fig:1_Zylinder_50mm}) und eines 
            \SI{75}{\milli\metre}-Resonators (siehe Abb.~\ref{fig:1_Zylinder_75mm}) mit dem Spektrum der Kette wechselnder Resonatorlänge in Abbildung~\ref{fig:1dim_10_Zylinder_wechselnd_50_75} zu erkennen.
            \FloatBarrier
            \begin{figure}[ht]
                \centering
                \begin{subfigure}[b]{0.45\textwidth}
                    \centering
                    \includegraphics[scale=0.45]{./pictures/1_Zylinder_50mm.pdf}
                    \caption{In der Abbildung ist das Frequenzspektrum eines einzelnen Zylinders der Länge \SI{75}{\milli\metre} dargestellt.}
                    \label{fig:1_Zylinder_50mm}
                \end{subfigure}
                \hfill
                \centering
                \begin{subfigure}[b]{0.45\textwidth}
                    \centering
                    \includegraphics[scale=0.45]{./pictures/1_Zylinder_75mm.pdf}
                    \caption{In der Abbildung ist das Frequenzspektrum eines einzelnen Zylinders der Länge \SI{75}{\milli\metre} dargestellt.}
                    \label{fig:1_Zylinder_75mm}
                \end{subfigure}
                \centering
                \begin{subfigure}[b]{0.45\textwidth}
                    \centering
                    \includegraphics[scale=0.45]{./pictures/1dim_10_Zylinder_wechselnd_50_75.pdf}
                    \caption{In der Grafik ist das Frequenzspektrum einer Kette aus zehn durch \SI{16}{\milli\metre}-Blenden getrennten Zylindern, die abwechselnd \SI{50}{\milli\metre} und \SI{75}{\milli\metre} lang sind, dargestellt.}
                    \label{fig:1dim_10_Zylinder_wechselnd_50_75}
                \end{subfigure}
                \caption{ }
                \label{fig:Wechselabbildung}
            \end{figure}
            \FloatBarrier
            
        \subsubsection*{Kette wechselnder Blendendurchmesser}
            Das bei wechselndem Blendendurchmesser von 13 und \SI{16}{\milli\metre} aufgenommene Frequenzspektrum ist in Abbildung~\ref{fig:1dim_8_Zylinder_blendewechselnd_13_16} dargestellt und zeigt eine 
            deutliche Aufspaltung der Maximabereiche um mindestens \SI{300}{\hertz}. Die neuen Bandlücken entstehen dadurch, dass sich das Potential nun alle zwei Zylinder periodisch wiederholt und nicht jeden 
            Zylinder. Dies entspricht einer neuen übergeordneten Periodizität des Gitters.  
            \begin{figure}[ht]
                \centering
                \includegraphics[scale=0.5]{./pictures/1dim_8_Zylinder_blendewechselnd_13_16.pdf}
                \caption{In der Abbildung ist das Frequenzspektrum einer Resonatorkette aus acht \SI{50}{\milli\metre} langen und abwechselnd durch \SI{13}{\milli\metre}- und \SI{16}{\milli\metre}-Blenden getrennten Zylindern dargestellt.}
                \label{fig:1dim_8_Zylinder_blendewechselnd_13_16}
            \end{figure}
            
            











































