\newpage
\section{Diskussion}
    Die Betrachtung der Analogons für die gegebenen Systeme des Wasserstoffatoms, des Wasserstoffmoleküls und eines 1-dimensionalen Festkörpers haben alle einen eindrucksvollen Einblick in die Grundlagen 
    dieser Systeme gegeben. Für das Wasserstoffatom konnten die Druckverteilungen weitesgehend eindeutig Atomorbitalen des Wasserstoffatoms zugeordnet werden. Die Aufspaltung der Resonanzen durch Aufhebung 
    der Symmetrieachse, sowie das Verhalten der Aufspaltung konnten besonders deutlich analysiert werden. Bei dieser Untersuchung konnten neben der l auch die m-Quantenzahl bestimmt werden.\newline
    Für das Wasserstoffmolekül ergab die Untersuchung der Resonanzfrequenz, dass die erster und dritter Ordnung unabhängig vom Durchmesser der Blende sind und demnach in der m-Quantenzahl entartet sind. Die 
    Resonanzfrequenz dritter Ordnung ist proportional zum Blendedurchmesser und nicht entartet. Aus den winkelabhängigen Druckamplitudenverteilungen ausgewählter Resonanzfrequenzen und den relativen 
    Phasenverschiebung derer zwischen oberen und unterem Resonator konnten die Molekülzustände für zwei der drei Resonanzen eindeutig bestimmt werden. Eine Resonanz konnte nicht vermessen werden.\newline
    Für den 1-dimensionalen-Festkörper konnten zunächst die grundlegenden Eigenschaften, wie das Entstehen von Bändern und Bandlücken sowie das Verhalten derer bei Änderung des auf das Elektron wirkenden
    Potentials überprüft werden. Anschließend zeigten die Störzylinder, wie die Bandstruktur durch Defekte innerhalb eines Gitters stark beeinflusst wird. Auch die Überlagerung der Bandstruktur zweier Basen 
    bei einer zwei-atomigen Basis konnte direkt durch eine Resonatorkette aus Resonatoren wechselnder Längen beobachtet werden. Zuletzt konnte auch der Einfluss einer übergeordneten Periodizität des Gitters 
    beobachtet werden, der darin liegt, dass neue Bandlücken entstehen.  