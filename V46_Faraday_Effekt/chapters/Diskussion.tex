%\newpage
\section{Diskussion}
    Der Versuch eignet sich zur ungefähren Bestimmung der effektiven Masse in GaAs, da die Abweichungen vom Theoriewert den Umständen gegeben relativ klein sind.

    Es ist zu erkennen, dass vereinzelt Werte nicht in das Gesamtbild passen. Das kann z.B. daran liegen, dass die Filter teilweise durch Fingerabdrücke verschmutzt waren oder auch, dass das Ablesen des Minimums mit einer relativ großen Unsicherheit verbunden ist, aufgrund dessen, weil die feinste Verstellung am Goniometer eine starke Verschiebung an den steilen Flanken im Oszilloskop bewirkt.
    Damit verbunden spielt der Ablesefehler des Winkels am Goniometer eine große Rolle.

    Es ist auch nicht auszuschließen, dass Schwankungen im Oszilloskop fälschlicherweise als 1. Minima interpretiert worden sind oder andersherum nicht als solche erkannt und als Schwankungen abgetan wurden.