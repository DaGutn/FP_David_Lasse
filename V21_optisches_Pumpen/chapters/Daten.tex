\newpage
\section{Daten}
    \begin{table}[h]
        \centering
        \caption{In dieser Tabelle sind die Spannungen der horizontalen Helmholtzspule und die der Sweep-Spule für alle RF-Frequenzen und beide Rubidium-Isotope aufgelistet. Auch die gesamte horizontalen Magnetfeldstärke, die sich aus den Spannungen berechnet, ist für alle RF-Frequenzen aufgetragen.}
        \label{tab:daten}
        \begin{tabular}{c c c c c c}
        \toprule
        {RF-Freq. [$\si{\kilo\hertz}]$} & {$\text{U}_{{}^{87, \, 85}\text{Rb}, \, \text{hor}}$ [V]} & {$\text{U}_{{}^{87}\text{Rb},\, \text{sweep}}$ [V]} & {$\text{U}_{{}^{85}\text{Rb},\, \text{hor}}$ [V]} & {$\text{B}_{{}^{87}\text{Rb}, \, \text{hor}}$ [\si{\micro\tesla}]}& {$\text{B}_{{}^{85}\text{Rb}, \, \text{hor}}$ [\si{\micro\tesla}]} \\
        \midrule
        100	 &  \num{13.84(2)}  &  \num{6.33(1)}   &  \num{7.50(1)}  & \num{48.72(526)}  & \num{55.78(526)} \\
        200	 &  \num{13.94(2)}  &  \num{4.38(1)}   &  \num{6.77(1)}  & \num{63.26(526)}  & \num{77.69(526)} \\
        300	 &  \num{13.94(2)}  &  \num{6.77(1)}   &  \num{10.27(1)} & \num{77.69(526)}  & \num{98.81(526)} \\
        400	 &  \num{14.04(2)}  &  \num{4.53(1)}   &  \num{9.25(1)}  & \num{90.48(526)}  & \num{118.96(526)} \\
        500	 &  \num{14.14(2)}  &  \num{2.45(1)}   &  \num{8.40(1)}  & \num{104.23(526)} & \num{140.14(526)} \\
        600	 &  \num{14.18(2)}  &  \num{2.87(1)}   &  \num{9.95(1)}  & \num{117.29(526)} & \num{160.02(526)} \\
        700	 &  \num{14.26(2)}  &  \num{1.89(1)}   &  \num{10.15(1)} & \num{132.43(526)} & \num{182.27(526)} \\
        800	 &  \num{14.30(2)}  &  \num{2.47(1)}   &  \num{10.25(1)} & \num{146.45(526)} & \num{203.92(526)} \\
        900	 &  \num{14.34(2)}  &  \num{3.18(1)}   &  \num{9.34(1)}  & \num{161.26(526)} & \num{224.74(526)} \\
        1000 &  \num{14.40(2)}  &  \num{2.62(1)}   &  \num{8.51(1)}  & \num{173.67(526)} & \num{246.04(526)} \\                  
        \bottomrule
        \end{tabular}
    \end{table}

    \FloatBarrier
