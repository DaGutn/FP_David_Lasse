\newpage
\section{Diskussion}
    Die bestimmten Kernspins der Rubidium-Isotope stimmen sehr gut mit den theoretischen Werten von $\text{I}_{\text{${}^{85}$Rb}} = \num{2.5}$ und 
    $\text{I}_{\text{${}^{87}$Rb}} = \num{1.5}$~\cite{millman_nuclear_1936} überein und weichen maximal um 3,6\% von diesen ab. \newline
    
    Die Abschätzung der Energieaufspaltung des quadratischen Zeemann-Effekts ergab Werte in der Größenordnung von maximal \SI{10e-12}{\electronvolt}. Dies entspricht 
    der Erwartung, dass der quadratische Zeemann-Effekt bei den maximal vorhandenen magnetischen Feldstärken im Bereich um \SI{250}{\micro\tesla} vernachlässigt werden kann. \newline
    
    In der Natur tritt ${}^{85}$Rb zu circa 72\% und ${}^{87}$Rb zu circa 28\% auf~\cite{noauthor_webelements_nodate}. In dem genutzten Gasgemisch hingegen ist in etwa doppelt so viel ${}^{85}$Rb wie ${}^{87}$Rb vorhanden. Diese Abweichung vom
    natürlichen Verhältnis ist jedoch absichtlich gewählt und korrekt bestimmt worden.\newline

    Bei der Bestimmung der Erdmagnetfeldstärke ist zunächst festzulegen, dass beide Werte mit circa \SI{35}{\micro\tesla} in der korrekten Größenordnung~\cite{noauthor_erdmagnetfeld_2021} liegen. Jedoch ist 
    der vertikale Anteil des Erdmagnetfelds in Europa für gewöhnlich stärker als der horizontale Anteil. Der vertikale Anteil wurde zu niedrig bestimmt und der vertikale Anteil zu hoch. 
    Die Abweichung des horizontalen Feldes könnte damit begründet werden, dass die horizontale Spule trotz Justierung nicht parallel genug zum horizontalen Erdmagnetfelds ausgerichtet gewesen ist. 
    Die Abweichung der vertikalen Komponente ist nicht über eine ungenaue Justierung der Spule, sondern womöglich durch externe Störfelder, die durch andere Experimente im Gebäude durchgeführt werden 
    zu begründen.    
