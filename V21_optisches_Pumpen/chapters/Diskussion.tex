\newpage
\section{Diskussion}
    Die Abweichungen der gemessenen Halbwertszeiten von den Literaturwerten sind beim Vanadium relativ gering.
    Allerdings ist die Abweichung der Halbwertszeit, die mit der vermeintlich verbesserten Rechnung bestimmt wurde größer als die die mit der Ausgleichsrechnung über alle Werte bestimmt wurde.

    Das liegt nur daran, dass die die Messwerte zufällig so verteilt sind, dass sich so eine Steigung ergibt. Normalerweise würde die zweite Ausgleichsrechnung ein Resultat ergeben, dass näher am tatsächlichen Wert liegt als die Erste. \\

    Beim Rhodium sind die Abweichungen der Halbwertszeit des langsamen Zerfalls natürlicherweise größer, da $t^*$ zum Einen nur per Augenmaß gewählt wird und die Werte in der zweiten Hälfte des Graphen relativ weit gestreut sind.

    Da die Werte für den schnellen Zerfall durch Subtraktion der Werte für den langsamen Zerfall entstehen, fließt die Ungenauigkeit der Ausgleichsrechnung für den langsamen Zerfall auch in die Berechnung der Halbwertszeit des schnellen Zerfalls mit ein und die Abweichung ist auch hier im Vergleich zum Vanadium relativ groß.
