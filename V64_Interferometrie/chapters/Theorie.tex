\section{Theoretische Grundlagen}
    \subsection{Interferenz und Kohärenz}
        Die Überlagerung von Wellen anhand des Superpositionsprinzips wird Interferenz genannt. Dabei ist es wichtig zu wissen ob die Wellen kohärent sind.

        Vollständig kohärent sind zwei Wellen, wenn ihre Phasendifferenz sowohl zeitlich als auch räumlich konstant bleibt.
        Unter zeitlicher Kohärenz ist zu verstehen, dass der Phasenunterschied in Ausbreitungsrichtung konstant ist.
        Räumliche Kohärenz heißt, dass der Phasenunterschied senkrecht zur Ausbreitungsrichtung konstant ist, also dass die Strahlen von den Lichtquellen zum betrachteten Ort parallel zueinander sind.

        Natürlich ist vollständige Kohärenz zwischen Wellen aus zwei verschiedenen Quellen in der Realität nicht erreichbar, da es keine reinen monochromatischen sondern nur quasimonochromatische Lichtquellen mit einer endlichen Kohärenzzeit $\Delta t_c$ gibt, und der Begriff der Teilkohärenz wird in diesem Zusammenhang verwendet.
        Teilkohärente Wellen führen zu einem stationären Interferenzbild.

        Inkohärente Wellen, also Wellen deren Phasendifferenz sich im betrachteten Bereich stark ändert, gleichen sich bei Interferenz zu einer mittleren Intensität aus.

    \subsection{Polarisation und Fresnel-Arago-Gesetze}
        Unpolarisiertes Licht kann immer in zwei zueinander senkrechte Komponenten zerlegt werden ($\vec{E} = \vec{E_{\bot}} + \vec{E_{\parallel}}$). Das selbe gilt für linear, zirkular und elliptisch polarisiertes Licht. Der einzige Unterschied zwischen diesen drei Polarisationen ist der Phasenunterschied $\delta$ zwischen den beiden Komponenten.
        \begin{itemize}
            \item Linear pol. Licht: $\qquad \quad \; \delta = 2\pi \cdot n, \qquad n \in Z$
            \item Zirkular pol. Licht: $\qquad \; \; \delta = \frac{\pi}{2} \cdot n, \qquad \, n \in Z \backslash \{0\}$
            \item Elliptisch pol. Licht: $\qquad$für jedes andere $\delta$
        \end{itemize}
        Das Interfenzmuster besteht also aus zwei einander überlagerten Interfenzmustern $\langle\left(\vec{E}_{1\bot} + \vec{E}_{2\bot}\right)^2\rangle_{\text{T}}$ und $\langle\left(\vec{E}_{1\parallel} + \vec{E}_{2\parallel}\right)^2\rangle_{\text{T}}$.
        Damit sind die Fresnel-Arago-Gesetze, also die Bedingungen dafür dass polarisiertes Licht interferiert, anhand der vektoriellen Formel
        \begin{align}
            I &= \langle\left(\vec{E}_1 + \vec{E}_2\right)^2\rangle_{\text{T}} \nonumber\\
            &= \langle\vec{E}^2_1\rangle_{\text{T}} + \langle\vec{E}^2_2\rangle_{\text{T}} + 2\langle\vec{E}_1\vec{E}_2\rangle_{\text{T}} \cos(\delta) \\
            &= I_1 + I_2 + I_{12}
            \label{eqn:interferenz}
        \end{align}
        leicht abzuleiten, wobei $\langle\rangle_{\text{T}}$ den zeitlichen Mittelwert kennzeichnet.
        \begin{enumerate}
            \item Zwei kohärente Wellen mit zueinander orthogonalen Auslenkungen (zwei im 90°-Winkel zueinander linear polarisierte Wellen) interferieren nicht, da in diesem Fall nach \eqref{eqn:interferenz} $I_{12} = 0$ ist.
            \item Natürliches Licht interferiert nicht, auch nicht wenn die Komponenten richtig ausgerichtet werden, da es inkohärent ist.
        \end{enumerate}
    
    \subsection{Interferometer}
        Die Sichtbarkeit bzw. der Kontrast $\mu$ eines Interferometers \cite{hecht_optik_2018} beschreibt die Qualität eines Interferenzmusters.
        Es quantifiziert wie stark sich die Maxima und Minima von der gemittelten gemeinsamen Intensität $I_1 + I_2$ abzeichnen.
        \begin{equation}
            \mu = \frac{I_{\text{max}} - I_{\text{min}}}{I_{\text{max}} + I_{\text{min}}} = \frac{2\langle\vec{E}_1\vec{E}_2\rangle_{\text{T}}}{\langle\vec{E}^2_1\rangle_{\text{T}} + \langle\vec{E}^2_2\rangle_{\text{T}}}
            \label{eqn:kontrast}
        \end{equation}
        Die Intensitäten $I_{\text{max}}$ und $I_{\text{min}}$ gehören dabei einem benachbarten Maximum und Minimum und lassen sich durch die folgende Formal bestimmen:
        \begin{equation}
            I_{\text{max/min}} \propto I_{\text{Laser}} \left(1 \pm 2 \cos(\phi)\sin(\phi)\right)
            \label{eqn:intensität_max_min}
        \end{equation}
        $I_{\text{Laser}}$ ist die mittlere Ausgangsintensität des Lasers und $\phi$ ist der Polarisationswinkel des Polarisators hinter dem Laser.
        Um \eqref{eqn:intensität_max_min} herzuleiten wird zuerst von der Gleichung
        \begin{equation}
            I \propto \langle\left(E_1 \cos(\omega t) + E_2 \cos(\omega t + \delta)\right)^2\rangle_{\text{T}}
            \label{eqn:intensität}
        \end{equation}
        ausgegangen. Für $E_1$ und $E_2$, die Amplituden des elektrischen Feldes der im Interferometer gegenläufigen Lichtstrahlen, mit dem Polarisationswinkel $\phi$ gilt
        \begin{equation*}
            E_1 = E \cos(\phi) \qquad \qquad E_2 = E \sin(\phi)
        \end{equation*}
        mit der Amplitude direkt nach dem Polarisator, aber vor Durchqueren des \textit{Polarizing Beam Splitter Cube} (PBSC), $E$.
        Werden die $E_i$ und das $\delta$ jeweils für konstruktive bzw. destruktive Interferenz in \eqref{eqn:intensität} eingesetzt, ergibt sich \eqref{eqn:intensität_max_min}.
        Die Abhängigkeit des Kontrastes vom Polarisationswinkel wird wiederum durch Einsetzten von \eqref{eqn:intensität_max_min} in \eqref{eqn:kontrast} bestimmt:
        \begin{equation*}
            \mu = \frac{4\sin(\phi)\cos(\phi)}{2} = \sin(2\phi)
        \end{equation*}
        Es ist zu erkennen, dass der Kontrast für die Winkel 45° und 135° maximal ist.

    \subsection{Detektion des Interferenzsignals}
        Die Intensität der Lichtstrahlen wird mithilfe von Photodioden gemessen. Diese bestehen aus Halbleitermaterial, in welchem durch Photonen Elektronen aus dem Valenzband ins Leitungsband gehoben werden. Dieser Photostrom ist messbar und proportional zur Intensität des gemessenen Lichtes.

        Die Differenzspannungsmethode wird verwendet indem ein zweiter um 45° geneigter PBSC in den Stahlengang gestellt wird und an beiden auf zwei Dioden fallenden austretenden Strahlen nun Interferenz beobachtet werden kann.
        Die Differenz der Spannungssignale hat jeweil bei Interferenzmaxima und -minima Nulldurchläufe.
        Diese Flanken sind so steil, dass sich die Maxima und Minima dadurch leicht auslesen lassen.
        Außerdem hilft die Differenzspannungsmethode dabei äußere Einflüsse zu minimieren.

    \subsection{Brechungsindizes}
        \subsubsection{Glasplatte}
            Die Anzahl der Interferenzmaxima bzw. -minima ist gegeben durch $M = \frac{\Delta \phi}{2 \pi}$. Eine planparallele Platte mit Brechungsindex $n$ erzeugt für kleine Drehwinkel $\theta$ bei Durchlassen von Licht eine Phasenverschiebung von
            \begin{equation}
                \Delta \phi = \frac{2\pi}{\lambda_{\text{vac}}} T \left(\frac{n - 1}{2n} \theta^2 + \mathcal{O}\left(\theta^4\right)\right)
                \label{eqn:phi_platte}
            \end{equation}
            mit der Wellenlänge des Lichtes im Vakuum $\lambda_{\text{vac}}$ und der Dicke der Platte $T$. \cite{tu_dortmund_versuchsanleitung_2021-1}\\
            Einsetzen von $\Delta \phi = 2\pi \cdot M$ in \eqref{eqn:phi_platte} ergibt die folgende Formel für den Brechungsindex:
            \begin{equation}
                n = \left(1 - 2\frac{M \lambda_{\text{vac}}}{T \theta^2}\right)^{-1}
                \label{eqn:n_platte}
            \end{equation}

        \subsubsection{Gas und Lorentz-Lorenz-Gesetz}
            Eine Gaszelle der Länge $L$ erzeugt analog zur Platte eine Phasenverschiebung von
            \begin{equation}
                \Delta \phi = \frac{2\pi}{\lambda_{\text{vac}}} \Delta n L
                \label{eqn:n_gas}
            \end{equation}
            Auch hier gilt der Zusammenhang $\Delta \phi = 2\pi \cdot M$ zwischen der Anzahl der jeweiligen Extrema und der Phasenverschiebung.

            Das Lorentz-Lorenz-Gesetz verbindet den Brechungsindex von Gasen mit der vorherrschenden Tempuratur $T$ und dem Druck $p$ (im Bereich des Normaldruckes),
            \begin{equation}
                n \approx \sqrt{1 + \frac{3 A p}{R T}}
                \label{eqn:lorentz}
            \end{equation}
            wobei $R$ die allgemeine Gaskonstante und $A$ die Molbrechung ist.