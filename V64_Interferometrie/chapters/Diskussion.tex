\section{Diskussion}
    Zunächst wurde der experimentelle Aufbau für einen maximalen Kontrast zwischen Interferenzmaxima und -minima justiert. Dieser liegt mit \num{0.82} in einem Bereich, der es erlaubt die Maxima und Minima 
    deutlich zu unterscheiden.\newline
    Die Bestimmung des Brechungsindex von Glas ergibt einen Wert, der um circa 28\% vom Theoriewert von Quarzglas abweicht. Der Theoriewert ist womöglich nicht sehr repräsentiv, da nicht bekannt ist welche 
    Art von Glas bei dem Versuch vermessen wurde. Generell liegt der bestimmte Wert jedoch im Größenordnungsbereich von anderen Glasen und ist damit passend.\newline
    Der Brechungsindex von Luft bei Normalbedingungen wurde sehr genau bestimmt, sodass der Theoriewert im Unsicherheitsintervall des gemessenen Bereichs liegt.\newline
    Demnach können beide Messungen als erfolgreich bezeichnet werden.       
    
