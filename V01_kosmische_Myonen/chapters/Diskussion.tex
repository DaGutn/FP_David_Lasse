\newpage
\section{Diskussion}
    Der Graph zur Justierung der Verzörgerungsleitungen in \autoref{fig:Verzoergerung} sollte eigentlich ein Plateau statt eines Peaks aufweisen. Wir können das Auftreten des Peaks jedoch leider nicht erklären.

    Der weitere Versuch wurde mit der Justage der Verzörgerungsleitungen fortgeführt, die dem Messpunkt im Ursprung des Graphen zugeordnet ist. Der Fit der Gaußglocke bestätigt diese Wahl, da seine Symmetrieachse sehr nah am Ursprung liegt.

    Die Nähe des theoretisch berechneten Untergrundes zum tatsächlich gemessenen Untergrund dient als Bestätigung der Annahme der Poissonverteilung.

    Die resultierende Lebensdauer der Myonen weicht nur ca. $6,6 \%$ vom eigentlichen Theoriewert ab, was ein Indiz dafür ist, dass die Messung erfolgreich verlaufen ist.

    


