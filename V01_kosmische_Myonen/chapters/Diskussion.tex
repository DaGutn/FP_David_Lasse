\newpage
\section{Diskussion}
    Der Graph zur Justierung der Verzörgerungsleitungen in \autoref{fig:Verzoergerung} weist ein leichtes Plateau auf.
    Dieses schwach ausgeprägte Plateau kann möglicherweise darauf zurückgeführt werden, dass die Zerfälle nicht genau in der Mitte des Szintillators, sondern immer in verschiedenen Plätzen passieren.

    Die Verzörgerungsleitungen wurden anschließend justiert und bei den folgenden Messungen in der Einstellung belassen, die dem Messpunkt im Ursprung des Graphen zuzuordnen ist. Der Fit der Plateau-Funktion bestätigt diese Wahl, da die Position ihrer Symmetrieachse sehr nah am Ursprung liegt.

    Die resultierende Lebensdauer der Myonen weicht nur ca. $5,7 \;\%$ vom eigentlichen Theoriewert ab, was ein Indiz dafür ist, dass die Messung erfolgreich verlaufen ist und dass die Annahme der Poissonverteilung zur Berechnung des Untergrundes in \autoref{sec:untergrund} bestätigt werden kann.

    


