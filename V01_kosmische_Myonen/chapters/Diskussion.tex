\newpage
\section{Diskussion}
    Der Graph zur Justierung der Verzörgerungsleitungen in \autoref{fig:Verzoergerung} weist ein leichtes Plateau auf.
    Dieses Plateau ist, wie in der Durchführung beschrieben, auf die normierte Pulsdauer der Ausgangssignale der Diskriminatoren zurückzuführen und befindet sich im erwarteten Bereich von \SI{10}{\nano \second}. \\
    Da das Plateau nicht sehr stark ausgeprägt ist, wurde eine Gaußfunktion höherer Potenz, anstatt einer Rechteckfunktion als Anpassungsfunktion gewählt.
    Dementsprechend ist die Halbwertsbreite der angepassten Funktion mit $T_{\text{FWHM}} = \SI{16,4}{\nano \second}$ größer als das eigentliche Plateau.

    Die bestimmte Lebensdauer der Myonen von $\tau_{\mu} = (2,072 \pm 0,0118) \, \mu\text{s}$ weicht nur ca. $5,7 \;\%$ vom Theoriewert $\tau_{\text{theo}} \approx 2,197 \, \mu$s \cite{zyla_review_2020} ab, was ein Indiz dafür ist, dass die Messung erfolgreich verlaufen ist und dass die Annahme der Poissonverteilung zur Berechnung des Untergrundes in \autoref{sec:untergrund} bestätigt werden kann.

    


