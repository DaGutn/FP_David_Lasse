\section{Einleitung}
    Die 1936 entdeckten Myonen gehören zu den Elementarteilchen des Standardmodells. Sie treten auf der Erdoberfläche in großer Menge auf, da sie ein Bestandteil der kosmischen Strahlung sind, die stets auf die 
    Erde trifft. Das Vorhandensein dieser sogenannten kosmischen Myonen und das exponentielle Verhalten von Teilchenzerfällen sollen genutzt werden, um die Lebensdauer dieses Elementarteilchens zu bestimmen. 
    Dazu wird ein Szintillatortank genutzt, der über mehrere Tage die Zeit zwischen Eintritt eines kosmischen Myons in den Tank und dessen Zerfall innerhalb des Tanks misst.      